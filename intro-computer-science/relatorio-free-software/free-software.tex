\documentclass{article}
\usepackage[utf8]{inputenc}

\title{Relatório Free Software}
\author{Henrique Leite}
\date{Fevereiro, 2023}

\begin{document}
\maketitle

\section{Free Software}
\paragraph{Free Software é um termo utilizado por aqueles que acreditam na liberdade de
explorar e conhecer por completo um determinado software. Este movimento social
baseia-se na liberdade de execução, cópia, distribuição, estudo, modificação e
melhoria contínua dos programas. As liberdades essenciais incluem:}

\begin{itemize}
    \item Liberdade para executar o software para qualquer fim;
    
    \item Liberdade para estudar e modificar o software para fazer o que se desejar;
    
    \item Liberdade para redistribuir cópias;

    \item Liberdade para distribuir cópias de versões modificadas.
\end{itemize}

\paragraph{Acesso ao código fonte é uma pré-condição para as liberdades 1 e 3. Legalmente, o Free Software é distribuído através de uma licença que respeita as 4 liberdades essenciais.}

\section{Fundador}
\paragraph{O fundador deste movimento é Richard M. Stallman, graduado em Física pela Universidade de Harvard e programador no laboratório de IA do MIT. Ele é conhecido por suas contribuições ao EMACS, ferramentas GNU, e licença GPL, que é a licença mais utilizada no mundo e representa o conceito de copyleft. Stallman dedica-se ao ativismo do Free Software desde os anos 90.}

\section{Open Source}
\paragraph{Outro termo que tornou-se popular é o "Open Source", que não tem foco em ideologia, mas é usado como uma "campanha de marketing" do Free Software. Empresas como Google, IBM, Oracle, Apple, Microsoft possuem softwares open source, motivadas pelos benefícios econômicos que este movimento pode trazer}

\section{Sociedade}
\paragraph{O movimento Free Software tem crescido ao longo dos anos e tem um impacto direto na vida da sociedade. É uma ponte com as questões éticas e morais do mundo digital e oferece uma boa perspectiva para o envolvimento social nas questões relacionadas ao software e usuários. Além disso, é importante destacar que a utilização de software livre também ajuda a preservar a privacidade dos usuários, já que eles têm acesso e controle sobre o funcionamento do software que estão usando.}

\end{document}